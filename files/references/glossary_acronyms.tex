% Acronyms
% \newacronym{api}{API}{Application Program Interface}

% Glossary
%\newglossaryentry{apig}{
%    name={API},
%    text={Application Program Interface},
%    sort=api,
%    description={In informatics, an API is a set of procedures available to programmers, typically grouped to form a toolkit for a specific task within a program. Its purpose is to provide an abstraction, usually between hardware and the programmer or between low-level and high-level software, simplifying the programming process}
%}

\newglossaryentry{machinelearning}{
    name={Machine Learning},
    text=machine learning,
    description={Con \textit{machine learning} si intende una branca dell'intelligenza artificale che si occupa di sviluppare algoritmi e modelli statistici per permettere ai computer di apprendere da dati e migliorare le proprie prestazioni, questo senza programmare istruzioni specifiche}
}

\newglossaryentry{deeplearning}{
    name=Deep Learning,
    text=deep learning,
    description={Con \textit{deep learning} si intende una sotto-categoria del \textit{machine learning}, che utilizza reti neurali artificiali che possiedono molteplici strati con lo scopo di analizzare e interpretare grandi moli di dati}
}

\newglossaryentry{GPU}{
    name=GPU,
    text=GPU,
    description={L'acronimo GPU (\textit{Graphics Processing Unit}) indica un componente \textit{hardware} creato specificatamente per elaborare immagini e per il rendering grafico. Sono progettate per gestire moltissime operazioni contemporaneamente, tramite l'uso di specifiche librerie è possibile sfruttare questa caratteristica per l'ambito dell'intelligenza artificiale}
}

\newglossaryentry{TPU}{
    name=TPU,
    text=TPU,
    description={L'acronimo TPU (\textit{Tensor Processing Unit}) indica un componente \textit{hardware} progettato specificatamente per il calcolo di moltissime operazioni matematiche in contemporanea. Esse sono progettate e prodotte specificatamente per essere utilizzate nell'ambito del machine learning}
}

\newglossaryentry{pullrequest}{
    name=Pull Request,
    text=pull request,
    description={Sono una funzionalità generalmente trovata nei sistemi di controllo di versione, consente agli sviluppatori di proporre modifiche ad una sorgente di codice. Questa funzionalità facilità la collaborazione, permettendo la revisione e discussione di modifiche prima che esse vengano integrate}
}

\newglossaryentry{computervision}{
    name=Computer Vision,
    text=computer vision,
    description={La \textit{computer vision} è un campo dell'intelligenza artificiale che si occupa di sviluppare sistemi in grado di interpretare e comprendere informazioni visive provenienti da immagini e video}
}

\newglossaryentry{dataset}{
    name=Dataset,
    text=dataset,
    description={Un \textit{dataset} è una collezione strutturata di dati, generalmente di grandi dimensioni, organizzata in forma relazionale}
}

\newglossaryentry{environment}{
    name=Environment,
    text=environment,
    description={Gli \textit{environment}, anche detti \textit{virtual environment}, sono spazi dedicati per l'installazione di specifici pacchetti e le loro dipendenze, senza influenzare altri progetti o l'intero sistema}
}

\newglossaryentry{neuralnetwork}{
    name=Rete Neurale Artificiale,
    text=rete neurale artificiale,
    description={Una rete neurale artificiale è un modello computazionale che si ispira alla struttura e al funzionamento del cervello umano. Questi modelli sono formati da unità di base che si connettono tra di loro per formare una rete. Le unità di base utilizzate si ispirano ad una versione parecchio semplificata dei neuroni biologici}
}

\newglossaryentry{ram}{
    name=RAM,
    text=RAM,
    description={La \textit{Random Access Memory} è un componente presente nei \textit{computer} che permette di immagazzinare dati che vengono letti e scritti, all'interno di essa, in modo rapido. Questa componente è necessaria per far funzionare programmi e per processare dati}
}

\newglossaryentry{api}{
    name=API,
    text=API,
    description={Un \textit{Application Programming Interface} è un \textit{set} di regole e protocolli che permettono a diverse applicazioni software di communicare tra di loro}
}

\newglossaryentry{preprocessing}{
    name=Preprocessing,
    text=preprocessing,
    description={Il \textit{preprocessing} è uno step che spesso viene effettuato quando si effettua analisi di dati o applicazione di \textit{machine learning}. In questo step si ha la trasformazione dei dati grezzi in una forma processata che è più adatta}
}

\newglossaryentry{downsampling}{
    name=Downsampling,
    text=downsampling,
    description={Il \textit{downsampling} è una tecnica di processamento dei dati che va a ridurre il numero di dati in un ``set'', ciò rende più semplice la gestione. Questo processo viene effettuato seguendo specifici metodi, che possono andare a causare la perdita di dati importanti se il livello di \textit{downsampling} è troppo severo}
}

\newglossaryentry{confusionmatrix}{
    name=Matrice di Confusione,
    text=matrice di confusione,
    description={La matrice di confusione è uno ``strumento'' utilizzato per valutare modelli di classificazione. Generalmente si tratta di una tabella quadrata dove vengono rappresentati i risultati delle predizioni del modello in confronto al vero valore}
}

\newglossaryentry{confidence}{
    name=Confidenza,
    text=confidenza,
    description={Un grafico della confidenza è una rappresentazione visuale che utilizza gli intervalli di confidenza per mostrare l'incertezza associata a stime statistiche. In questo ambito viene usato per capire quanto un modello di classificazione è sicuro delle predizioni che fa}
}

\newglossaryentry{epoch}{
    name=Epoca,
    text=epoca,
    description={Un'epoca, nel contesto del \textit{machine learning}, è una singola iterazione della fase di addestramento dove viene percorso l'intero \textit{training dataset}}
}

\newglossaryentry{prefetch}{
    name=Prefetch,
    text=prefetch,
    description={Il \textit{prefetch} è una tecnica utilizzata per caricare anticipatamente dei dati che verranno utilizzati a breve}
}

\newglossaryentry{label}{
    name=Label,
    text=label,
    description={Una \textit{label}, anche detta etichetta, è un'annotazione che assegna una categoria ad uno specifico dato. Vengono utilizzate confrontare il risultato ottenuto da un modello con l'effettiva categoria}
}

\newglossaryentry{dicom}{
    name=DICOM,
    text=DICOM,
    description={DICOM (\textit{Digital Imaging and Communications in Medicine}) è uno standard usato in campo medico per l'immagazzinamento, transmissione e gestione delle immagini medicali}
}

\newglossaryentry{batch}{
    name=Batch,
    text=batch,
    description={Con il termine \textit{batch}, si definisce una porzione di \textit{dataset} che viene utilizzata per essere processata in un'insieme di dati in una singola iterazione di addestramento. È un punto chiave di un processo di ottimizzazione per fare in modo di processare in un modo più efficiente il \textit{dataset}}
}