% Acronyms
\newacronym{api}{API}{Application Program Interface}
\newacronym{sdk}{SDK}{Software Development Kit}
\newacronym{uml}{UML}{Unified Modeling Language}
\newacronym{tsa}{TSA}{Termine solo acronimo}

% Glossary
\newglossaryentry{apig}{
    name={API},
    text={Application Program Interface},
    sort=api,
    description={In informatics, an API is a set of procedures available to programmers, typically grouped to form a toolkit for a specific task within a program. Its purpose is to provide an abstraction, usually between hardware and the programmer or between low-level and high-level software, simplifying the programming process}
}

\newglossaryentry{sdkg}{
    name={SDK},
    text={Software Development Kit},
    sort=sdk,
    description={A Software Development Kit (SDK) is a collection of development tools in one installable package, facilitating application creation by providing a compiler, debugger, and sometimes a software framework. SDKs are typically specific to a hardware platform and operating system combination. Many application developers use specific SDKs to enable advanced functionalities such as advertisements, push notifications, etc}
}

\newglossaryentry{umlg}{
    name={UML},
    text={Unified Modeling Language},
    sort=uml,
    description={In software engineering, Unified Modeling Language (UML) is a modeling and specification language based on the object-oriented paradigm. UML serves as a "lingua franca" in the object-oriented design and programming community. Much of the industry literature uses UML to describe analytical and design solutions in a concise and understandable way for a broad audience}
}

\newglossaryentry{machinelearning}{
    name={Machine Learning},
    text=machine learning,
    description={Con \textit{machine learning} si intende una branca dell'intelligenza artificale che si occupa di sviluppare algoritmi e modelli statistici per permettere ai computer di apprendere da dati e migliorare le proprie prestazioni, questo senza programmare istruzioni specifiche}
}

\newglossaryentry{deeplearning}{
    name=Deep Learning,
    text=deep learning,
    description={Con \textit{deep learning} si intende una sotto-categoria del \textit{machine learning}, che utilizza reti neurali artificiali che possiedono molteplici strati con lo scopo di analizzare e interpretare grandi moli di dati}
}

\newglossaryentry{GPU}{
    name=GPU,
    text=GPU,
    description={L'acronimo GPU (\textit{Graphics Processing Unit}) indica un componente \textit{hardware} creato specificatamente per elaborare immagini e per il rendering grafico. Sono progettate per gestire moltissime operazioni contemporaneamente, tramite l'uso di specifiche librerie è possibile sfruttare questa caratteristica per l'ambito dell'intelligenza artificale}
}

\newglossaryentry{TPU}{
    name=TPU,
    text=TPU,
    description={L'acronimo TPU (\textit{Tensor Processing Unit}) indica un componente \textit{hardware} progettato specificatamente per il calcolo di moltissime operazioni matematiche in contemporanea. Esse sono progettate e prodotte specificatamente per essere utilizzate nell'ambito del machine learning}
}

\newglossaryentry{pullrequest}{
    name=Pull Request,
    text=pull request,
    description={Sono una funzionalità generalmente trovata nei sistemi di controllo di versione, consente agli sviluppatori di proporre modifiche ad una sorgente di codice. Questa funzionalità facilità la collaborazione, permettendo la revisione e discussione di modifiche prima che esse vengano integrate}
}

\newglossaryentry{computervision}{
    name=Computer Vision,
    text=computer vision,
    description={La \textit{computer vision} è un campo dell'intelligenza artificale che si occupa di sviluppare sistemi in grado di interpretare e comprendere informazioni visive provenienti da immagini e video}
}

\newglossaryentry{dataset}{
    name=Dataset,
    text=dataset,
    description={Un \textit{dataset} è una collezione strutturata di dati, generalmente di grandi dimensioni, organizzata in forma relazionale.}
}

\newglossaryentry{environment}{
    name=Environment,
    text=environment,
    description={Gli \textit{environment}, anche detti \textit{virtual environment}, sono spazi dedicati per l'installazione di specifici pacchetti e le loro dipendenze, senza influenzare altri progetti o l'intero sistema}
}