% Load variables
\newcommand{\myUni}{Università degli Studi di Padova}
\newcommand{\myDepartment}{Dipartimento di Matematica ``Tullio Levi-Civita''}
\newcommand{\myFaculty}{Corso di Laurea in Informatica}
\newcommand{\myTitle}{Utilizzo di Deep Learning per il
rilevamento di anomalie elettrocardiografiche}
\newcommand{\myDegree}{Tesi di Laurea Triennale}
\newcommand{\profTitle}{Prof.}
\newcommand{\myProf}{Tullio Vardanega}
\newcommand{\graduateTitle}{Laureando}
\newcommand{\myName}{Oscar Konieczny}
\newcommand{\myStudentID}{2042335}
\newcommand{\myAA}{2023-2024}
\newcommand{\myLocation}{Padova}
\newcommand{\myTime}{Dicembre 2024}
% Acronyms
\newacronym{api}{API}{Application Program Interface}

% Glossary
\newglossaryentry{apig}{
    name={API},
    text={Application Program Interface},
    sort=api,
    description={In informatics, an API is a set of procedures available to programmers, typically grouped to form a toolkit for a specific task within a program. Its purpose is to provide an abstraction, usually between hardware and the programmer or between low-level and high-level software, simplifying the programming process}
}

\newglossaryentry{machinelearning}{
    name={Machine Learning},
    text=machine learning,
    description={Con \textit{machine learning} si intende una branca dell'intelligenza artificale che si occupa di sviluppare algoritmi e modelli statistici per permettere ai computer di apprendere da dati e migliorare le proprie prestazioni, questo senza programmare istruzioni specifiche}
}

\newglossaryentry{deeplearning}{
    name=Deep Learning,
    text=deep learning,
    description={Con \textit{deep learning} si intende una sotto-categoria del \textit{machine learning}, che utilizza reti neurali artificiali che possiedono molteplici strati con lo scopo di analizzare e interpretare grandi moli di dati}
}

\newglossaryentry{GPU}{
    name=GPU,
    text=GPU,
    description={L'acronimo GPU (\textit{Graphics Processing Unit}) indica un componente \textit{hardware} creato specificatamente per elaborare immagini e per il rendering grafico. Sono progettate per gestire moltissime operazioni contemporaneamente, tramite l'uso di specifiche librerie è possibile sfruttare questa caratteristica per l'ambito dell'intelligenza artificiale}
}

\newglossaryentry{TPU}{
    name=TPU,
    text=TPU,
    description={L'acronimo TPU (\textit{Tensor Processing Unit}) indica un componente \textit{hardware} progettato specificatamente per il calcolo di moltissime operazioni matematiche in contemporanea. Esse sono progettate e prodotte specificatamente per essere utilizzate nell'ambito del machine learning}
}

\newglossaryentry{pullrequest}{
    name=Pull Request,
    text=pull request,
    description={Sono una funzionalità generalmente trovata nei sistemi di controllo di versione, consente agli sviluppatori di proporre modifiche ad una sorgente di codice. Questa funzionalità facilità la collaborazione, permettendo la revisione e discussione di modifiche prima che esse vengano integrate}
}

\newglossaryentry{computervision}{
    name=Computer Vision,
    text=computer vision,
    description={La \textit{computer vision} è un campo dell'intelligenza artificiale che si occupa di sviluppare sistemi in grado di interpretare e comprendere informazioni visive provenienti da immagini e video}
}

\newglossaryentry{dataset}{
    name=Dataset,
    text=dataset,
    description={Un \textit{dataset} è una collezione strutturata di dati, generalmente di grandi dimensioni, organizzata in forma relazionale}
}

\newglossaryentry{environment}{
    name=Environment,
    text=environment,
    description={Gli \textit{environment}, anche detti \textit{virtual environment}, sono spazi dedicati per l'installazione di specifici pacchetti e le loro dipendenze, senza influenzare altri progetti o l'intero sistema}
}

\newglossaryentry{neuralnetwork}{
    name=Rete Neurale Artificiale,
    text=rete neurale artificiale,
    description={Una rete neurale artificiale è un modello computazionale che si ispira alla struttura e al funzionamento del cervello umano. Questi modelli sono formati da unità di base che si connettono tra di loro per formare una rete. Le unità di base utilizzate si ispirano ad una versione parecchio semplificata dei neuroni biologici}
}

% Define custom colors
\definecolor{hyperColor}{HTML}{0B3EE3}
\definecolor{tableGray}{HTML}{F5F5F7}
\definecolor{veryPeri}{HTML}{6667ab}

% Set line height
\linespread{1.5}

% Custom hyphenation rules
\hyphenation {
    data-base
    al-go-rithms
    soft-ware
}

% Images path
\graphicspath{{img/}}

% Force page color, as some editors set a grayish color as default
\pagecolor{white}

% Better spacing for footnotes
\setlength{\skip\footins}{5mm}
\setlength{\footnotesep}{5mm}

\setlength{\headheight}{14.5pt}
\addtolength{\topmargin}{-2.45pt}

% Add a subscript G to a glossary entry
\newcommand{\glox}{\textsubscript{\textbf{\textit{G}}}}

% Improvements the paragraph command
\titleformat{\paragraph}
{\normalfont\normalsize\bfseries}{\theparagraph}{1em}{}
\titlespacing*{\paragraph}
{0pt}{3.25ex plus 1ex minus .2ex}{1.5ex plus .2ex}

% Define use case environment
\newcounter{usecasecounter} % define a counter
\setcounter{usecasecounter}{0} % set the counter to some initial value
% Parameters
% #1: ID
% #2: Nome
\newenvironment{usecase}[2]{
    \renewcommand{\theusecasecounter}{\usecasename #1}  % this is where the display of the counter is overwritten/modified
    \refstepcounter{usecasecounter} % increment counter
    \vspace{2em}
    \par \noindent % start new paragraph
    {\normalsize \textbf{\usecasename #1: #2}} % display the title before the content of the environment is displayed
    \vspace{.5em}
}{
    \medskip
}
\newcommand{\usecasename}{UC}
\newcommand{\usecaseactors}[1]{\textbf{\\Attori Principali:} #1}
\newcommand{\usecasepre}[1]{\textbf{\\Precondizioni:} #1}
\newcommand{\usecasedesc}[1]{\textbf{\\Descrizione:} #1}
\newcommand{\usecasepost}[1]{\textbf{\\Postcondizioni:} #1}
\newcommand{\usecasealt}[1]{\textbf{\\Scenario Alternativo:} #1}

% Define risks environment
\newcounter{riskcounter} % define a counter
\setcounter{riskcounter}{0} % set the counter to some initial value
% Parameters
% #1: Title
\newenvironment{risk}[1]{
    \refstepcounter{riskcounter} % increment counter
    \par \noindent % start new paragraph
    \textbf{\arabic{riskcounter}. #1} % display the title before the content of the environment is displayed
}{
    \par\medskip
}
\newcommand{\riskname}{Rischio}
\newcommand{\riskdescription}[1]{\textbf{\\Descrizione:} #1.}
\newcommand{\risksolution}[1]{\textbf{\\Soluzione:} #1.}

% Apply fancy styling to pages
\pagestyle{fancy}
\fancyhf{}
\fancyhead[L]{\leftmark} % Places Chapter N. Chatper title on the top left
\fancyfoot[C]{\thepage} % Page number in the center of the footer

% Adds a blank page while increasing the page number
\newcommand\blankpage{ 
\clearpage
    \begingroup
    \null
    \thispagestyle{empty}
    \hypersetup{pageanchor=false}
    \clearpage
\endgroup
}

% Adds a blank page while increasing the page number
\newcommand\blankpagewithnumber{ 
  \clearpage
  \thispagestyle{plain} % Use plain page style to keep the page number
  \null
  \clearpage
}

% Increase page numbering
\newcommand\increasepagenumbering{
    \addtocounter{page}{+1}
}

% Make glossaries and bibliography
\makeglossaries
% Redefine the format for the glossary entries to be italic
\renewcommand*{\glstextformat}[1]{\textit{#1}\glox}
%\glsaddall

\bibliography{references/bibliography}
\defbibheading{bibliography} {
    \cleardoublepage
    \phantomsection
    \addcontentsline{toc}{chapter}{\bibname}
    \chapter*{\bibname\markboth{\bibname}{\bibname}}
}

% Code blocks handling w/ table of codes
\makeatletter
\ifdefined\NR@chapter
  \expandafter\@firstoftwo
\else
  \expandafter\@secondoftwo
\fi{\patchcmd\NR@chapter}{\patchcmd\@chapter}
  {\addtocontents{lot}{\protect\addvspace{10\p@}}}
  {\addtocontents{lot}{\protect\addvspace{10\p@}}%
   \addtocontents{lol}{\protect\addvspace{10\p@}}}
  {}{}
\makeatother

\renewcommand\listingscaption{Codice}
\renewcommand\listoflistingscaption{Elenco dei codici sorgenti}
\counterwithin*{listing}{chapter}
\renewcommand\thelisting{\thechapter.\arabic{listing}}

% Set up hyperlinks
\hypersetup{
    colorlinks=true,
    linktocpage=true,
    pdfstartpage=1,
    pdfstartview=,
    breaklinks=true,
    pdfpagemode=UseNone,
    pageanchor=true,
    pdfpagemode=UseOutlines,
    plainpages=false,
    bookmarksnumbered,
    bookmarksopen=true,
    bookmarksopenlevel=1,
    hypertexnames=true,
    pdfhighlight=/O,
    allcolors = hyperColor
}

% Set up captions
\captionsetup{
    tableposition=top,
    figureposition=bottom,
    font=small,
    format=hang,
    labelfont=bf
}

% When draft mode is on, the hyperlinks are removed. Useful when printing the document. To enable/disable, uncomment/comment the command
% \hypersetup{draft}

% prevents cleaning up the cache at the end of the run (needed to keep the unused caches, generated by other editions)
\makeatletter
\renewcommand*{\minted@cleancache}{}
\makeatother

% Break lines in code blocks whe using inputminted
\setminted{breaklines}

% Interline of the code parts
\setminted{baselinestretch=0.85}

% Highlight colors
\definecolor{lightred}{RGB}{255,230,230}
\definecolor{lightgreen}{RGB}{230,255,230}
\definecolor{lightblue}{RGB}{230,230,255}