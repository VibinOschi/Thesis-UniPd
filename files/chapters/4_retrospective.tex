\chapter{Retrospettiva}
\label{chap:retrospettiva}

\section{Conseguimento degli obiettivi}\label{sec:objectives-achieved}

\subsection{Obiettivi aziendali}\noindent
% \texttt{Esposizione degli obiettivi aziendali raggiunti, facendo riferimento agli obiettivi descritti nelle sezioni precedenti.}
\begin{center}
    \rowcolors{1}{}{tableGray}
    \begin{longtable}{|p{9.5cm}|p{2.5cm}|}
    \hline
    \multicolumn{1}{|c|}{\textbf{Obiettivo}} & \multicolumn{1}{c|}{\textbf{Stato}}\\ 
    \hline 
    \endfirsthead
    \rowcolor{white}
    \multicolumn{2}{c}{{\bfseries \tablename\ \thetable{} -- Continuo della tabella}}\\
    \hline
    \multicolumn{1}{|c|}{\textbf{A}} & \multicolumn{1}{c|}{B}\\ \hline 
    \endhead
    \hline
    \rowcolor{white}
    \multicolumn{2}{|r|}{{Continua nella prossima pagina...}}\\
    \hline
    \endfoot
    \endlastfoot 
    
   Studio delle tecnologie & \multicolumn{1}{c|}{Completato} \\
    \hline
    Creazione di un dataset appropriato e di tecniche per la sua gestione & \multicolumn{1}{c|}{Completato} \\
    \hline
    Implementazione in \textit{Python} degli algoritmi basati su tecniche di \textit{deep learning} & \multicolumn{1}{c|}{Completato} \\
    \hline
    Valutazione dell'esecuzione  degli algoritmi & \multicolumn{1}{c|}{Completato} \\
    \hline
    Realizzazione di \textit{unit test} & \multicolumn{1}{c|}{Parzialmente completato} \\
    \hline
    Idee o suggerimenti su come migliorare in futuro la performance del software & \multicolumn{1}{c|}{Completato} \\
    \hline
    \hiderowcolors
    \caption{Lista del completamento degli obiettivi aziendali.}
    \label{tab:res-obiettivi}
    \end{longtable}
\end{center}\vspace{-1cm}\noindent
La tabella \ref{tab:res-obiettivi} mostra i vari stati di completamento dei vari obiettivi che erano stati definiti precedentemente nella tabella \ref{tab:obiettivi}.\\
Ho raggiunto la maggior parte degli obiettivi, con l'eccezione dell'obiettivo della realizzazione dei \textit{test} di unità.
Non ho realizzato abbastanza \textit{test} di questo tipo per poter segnare questo obiettivo come raggiunto pienamente.
Questo, però, non è di per se una problematica, siccome questo obiettivo non era obbligatorio nella completazione.

\subsection{Obiettivi personali}\noindent
% \texttt{Esposizione degli obiettivi personali raggiunti, facendo riferimento agli obiettivi descritti nelle sezioni precedenti.}
Come menzionato nella sezione \ref{sec:choice-motivation}, prima di effettivamente iniziare lo \textit{stage} mi sono imposto diversi obiettivi personali.
\begin{itemize}
    \item \textbf{Approfondimento tecnico:} lavorando al progetto ho potuto apprendere nuove cose nell'ambito dell'\gls{machinelearning}. Ho potuto familiarizzare con la tecnologie di Keras e TensorFlow. Questo mi ha permesso di espandere le mie conoscenze sui vari tipi di reti neurali artificiali che vengono utilizzati in ambito di analisi di serie temporali. Inoltre ho imparato come vengono gestiti i \textit{dataset} di qualsiasi dimensione, indipendentemente della forma dei dati contenuti al suo interno.
    \item \textbf{Capire l'ambiente lavorativo:} svolgendo lo \textit{stage} ho potuto immergermi pienamente nel mondo del lavoro. In azienda ho potuto partecipare in prima persona ai vari metodi che caratterizzano lo sviluppo \textit{software}, mettendo in pratica quello che ho imparato nel corso universitario di ``Ingegneria del \textit{Software}''. Inoltre ho potuto capire la cultura lavorativa di un'azienda di questo settore, che era qualcosa di cui ero personalmente curioso, non avendo mai avuto una esperienza di questo tipo.
    \item \textbf{\textit{Problem solving}:} durante lo svolgimento di diverse attività, ho potuto affrontare diverse sfide e problematiche che si sono presentate. Grazie a ciò ho potuto migliorare le mie abilità nel \textit{problem solving}.
\end{itemize}

\newpage

\section{Competenze acquisite}\label{sec:expertise-gained}\noindent
% \texttt{Descrizione della maturazione professionale, in ambito di conoscenze e abilità.}
Le competenze che ho acquisito coincidono abbastanza con i miei obiettivi personali.
Posso categorizzare le competenze in due ambiti:
\begin{itemize}
    \item \textbf{Competenze tecniche:} svolgendo lo \textit{stage} ho potuto espandere la mia conoscenza e abilità in varie tecnologie e argomenti. Ho potuto appronfodire le mie conoscenze su Keras e su varie tipologie di reti neurali artificiali e ho imparato come vengono gestiti i \textit{dataset} nel contesto del \textit{machine learning}. Ma oltre alle tecnologie, ho potuto acquisire esperienza con la parte pratica della metodologia \textit{Agile}.
    \item \textbf{Competenze trasversali:} ho avuto varie opportunità per sviluppare sia \textit{soft skills} che il \textit{problem solving}. In parecchie occasioni ho potuto migliorare le mie capacità espressive e di presentazione, questo per via del fatto che ho svolto vari \textit{sprint review} e una presentazione in azienda, dove ho esposto la mia esperienza di \textit{stage}. Inoltre, ho affrontato diverse sfide che mi hanno permesso di migliorare le mie abilità di \textit{problem solving}.
\end{itemize}

\section{Divario tra università e lavoro}\label{sec:uni-work-diff}\noindent
% \texttt{Descrizione della distanza tra le competenze imparate in università e quelle richieste allo \textit{stage}.}
Con la mia esperienza di \textit{stage} in azienda ho potuto vivere in prima persona la differenza che è presente tra l'università e il mondo del lavoro.\\
Questi due ambienti differiscono tra di loro, ma possiedono comunque diversi punti in comune. L'università è più un luogo dove ci si incentra più su l'aspetto teorico delle cose e lo studente è sottoposto ad un rapporto che è principalmente unilaterale, dove esso principalmente riceve le informazioni e la produzione effettiva che svolge è poca.\\
Per un certo senso, il mondo di lavoro è opposto per questo aspetto. Tuttavia, sia l'università che il lavoro, possiedono in parte quello che caratterizza di più l'altro: l'università rende partecipi gli studenti a svolgere alcuni progetti che simulano, in forma o un'altra, l'effettiva parte pratica, mentre nel lavoro, se si vuole effettuare innovazione, ci si deve prima immergere nella teoria.\\
L'opportunità di \textit{stage} è un buon momento per valutare la preparazione che l'università offre per il mondo del lavoro. \\ Nel mio caso posso considerarmi alquanto soddisfatto delle conoscenze acquisite durante i miei anni in università.
Tra i corsi che ho seguito, menziono quelli che hanno avuto un impatto positivo per lo svolgimento del tirocinio. Uno di essi è ``Ingegneria del Software'': questo corso permette di vedere, sia in modo teorico, sia in modo pratico con un cospicuo progetto, le parti strutturali del lavoro che si effettua nelle varie aziende.\\
Oltre a imparare metodi lavorativi, alcuni corsi mi hanno dato una buona infarinatura sulla parte delle conoscenze, che nel caso del mio progetto in azienda, si basava principalmente sul \textit{machine learning}. I corsi in questione, che ho seguito, sono: ``Introduzione all'apprendimento automatico'' e ``Intelligenza Artificale'' del Dipartimento di Psicologia. Entrambi i corsi erano opzionali nel piano di studio, ma mi hanno dato buone basi teoriche che sono state supplementate con piccoli progetti, in gran parte simili a quello che ho svolto in azienda.
\newpage